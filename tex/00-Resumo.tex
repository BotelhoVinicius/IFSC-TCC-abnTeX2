\begin{resumo}
	Este trabalho apresenta o estudo de técnicas de compatibilidade eletromagnética para a mitigação do ruído conduzido e irradiado em um conversor estático do tipo Buck \Interleaved. O conversor proposto teve seu projeto iniciado na disciplina de Eletrônica de Potencia e apresenta uma  corrente de saída de \SI{5}{\ampere}, tensão de saída de \SI{5}{\volt} e potência nominal de 25 watts. Neste trabalho, foram realizados os ensaios de emissão conduzida e irradiada do conversor projetado, bem como os ensaios após a aplicação das técnicas propostas, aplicadas de forma isoladas. A fim de ser possível a comparação dos resultados dos ensaios, estabeleceu-se uma padronização dos ensaios, tanto em relação ao posicionamento da placa quanto ao do cabo. Dentre os testes propostos, alguns destacaram-se por apresentar resultados significativos. Uma grande solução apresentada foi a redução da frequência de chaveamento com o uso de um conversor com célula de comutação de três estados. As técnicas testadas de forma isolada não apresentaram melhoras significativas simultaneamente na emissão conduzida e irradiada, porém, ao serem utilizadas de forma combinada, mostraram-se de grande eficácia.
	
	\textbf{Palavras-chave}: Compatibilidade eletromagnética. Eletrônica de potência. Buck \interleaved. Conversor Buck com célula de comutação de três estados.
\end{resumo}